\documentclass[a4paper]{article}
\usepackage[dutch]{babel}
\usepackage{hyperref}
\usepackage{enumerate}
\usepackage{xspace}
\usepackage{cite}

\title{RDC4\\Bitcoins als wereldwijd betaalmiddel}
\author{Jelte Fennema - 10183159}
\date{\today}


\begin{document}
\maketitle
\begin{abstract}
    Mensen zijn op het moment gewend aan een munteenheid voor elk land en in het
    geval van de Euro voor een verzameling van landen. Als men dan in een ander
    land wil betalen moet men het geld wisselen. Op het wisselen wordt dan
    verlies gemaakt. Dit wordt al vele jaren zo gedaan, maar is het wel de beste
    manier?

    Per land de munteenheid regelen heeft natuurlijk voordelen, maar is een
    globale munteenheid niet handiger nu de wereld steeds globaler wordt? De
    opkomst van Bitcoins heeft ervoor gezorgd dat een globale munteenheid niet
    meer als iets onmogelijks klinkt.

\end{abstract}

\newpage
\tableofcontents
\newpage

\section{Introductie}
Nog niet zo heel lang geleden was het nog een hele tocht om van Amsterdam naar
Leeuwarden te gaan. Tegenwoordig ben je in een dagje van Amsterdam naar New
York. Deze globalisatie heeft invloed gehad op een heleboel onderdelen van het
dagelijks leven. Vreemde talen die geleerd worden, voedsel dat gegeten wordt,
vakanties waarop gegaan wordt en nog een heleboel meer.

E\'en van de dingen uit het dagelijks leven die maar langzaam mee gaat met de
globalisatie trend is de betaalwijze die mensen gebruiken. De betaalwijze van
mensen is uiteraard wel een heel stuk anders dan vroeger: Creditcards en
pinpassen worden tegenwoordig op veel plekken geaccepteerd, digitaal overmaken
kan ook gewoon naar een bankrekening aan de andere kant van de wereld en overal
is wel een wisselkantoor of bank te vinden als het echt niet anders kan.

Dat is echter niet het globaliseren van de betaalwijze die mensen gebruiken, dat
is het gebruiken van de oude betaalwijze op een manier waarvoor die niet bedoelt
is.  Het geld waarmee betaald wordt, waar die nummers op de rekening voor staan,
is nog steeds de munteenheid die gebonden is aan een land, of meerdere landen in
het geval van de Euro.

Waarom is er geen geglobaliseerde munteenheid? Zou het in deze tijd van
globalisering niet grote voordelen hebben om een munt te hebben die overal
geaccepteerd wordt, zonder dat er eerst door de bank een paar procent
wisselkoersopslag vanaf gehaald zou moeten worden om het om te vormen naar een
geaccepteerde munt? Zo'n munteenheid zou aan een aantal eisen moeten voldoen om
\"uberhaupt te kunnen bestaan en aan nog meer om ook goed te kunnen werken.
Bitcoin is een munteenheid die een op het moment een behoorlijk grijp doet naar
die titel en zou daadwerkelijk die munteenheid kunnen worden.

\subsection{Wat is Bitcoin?}
Helemaal in de basis is Bitcoin peer-to-peer, digitaal, contant geld.
\cite[p1]{nakamoto2008bitcoin} Deze drie onderdelen, zijn allemaal belangrijk
voor het succes van Bitcoin. Ze zullen verder los behandeld worden, maar er
zijn een paar dingen die handig zijn om van te voren even goed op een rijtje te
hebben. Doordat het peer-to-peer is, zijn er geen tussenpartijen nodig. Alleen
gebruikers hebben is genoeg om het hele systeem werkende te houden. Het is
contant geld, op de manier dat zodra een transactie is gedaan van \'e\'en
entiteit naar een andere, dat niet teruggedraaid kan worden. Kortom het is niet
zoals de meeste andere digitale betaalmethoden, zoals Paypal of iDeal.

\section{Problemen met een wereldwijde munteenheid}
Om te zorgen dat een munteenheid gebruikt kan worden als wereldwijde
munteenheid zijn er een aantal criteria waar aan voldaan moet worden.

De belangrijkste voorwaarde is dat het totaal onafhankelijk moet zijn. Er moet
geen enkele persoon, groep, bedrijf of overheid zijn die verantwoordelijk is
voor een bepaald onderdeel van de munteenheid. Zo'n onderdeel kan bijvoorbeeld
zijn transactie validatie, de gelduitgifte of het transactie vervoer. De reden
hiervoor is dat die entiteit zou kunnen stoppen met die taak uitvoeren, wat dus
het instorten van het systeem zou betekenen.

Verder moet de munteenheid toegankelijk zijn voor praktisch iedereen. Uiteraard
heeft een wereldwijde munteenheid weinig zin als er maar een aantal mensen zijn
die het kunnen gebruiken.

Als laatste moet de betaalmethode veilig zijn. Er moet voor iedereen
inzichtelijk zijn wie wat aan wie betaalt. Dit wordt bij huidige systemen door
een bank of tussenpartij geregeld. Zoals eerder gesteld kan is dat dus geen
optie voor een wereldwijde munteenheid.

In de volgende onderdelen zal uitgelegd worden hoe Bitcoin deze problemen oplost
en daarbij nog een aantal andere voordelen heeft ten opzichte van de huidige
betaalmethoden.


\section{Openheid van Bitcoin}

\section{Anonimiteit van Bitcoin}

\section{Decentralisatie van Bitcoin}

\section{Het digitaal zijn van Bitcoin}

\subsection{Fysiek betalen met Bitcoin}

\section{Prijs voor het gebruik van Bitcoin}

\section{Conclusie}


\newpage
\bibliography{essay}
\bibliographystyle{plain}

\end{document}
